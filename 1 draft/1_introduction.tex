% !TEX root = EIMW2022.tex

\section{Introduction\label{SECTION: Introduction}}

In light of historically high levels of income inequality in many places, understanding the effects of labor market policies on the distribution of income and employment is seen as increasingly important. Several countries have recently implemented higher minimum wages in an attempt to aid low-income workers. Yet the benefits and costs of minimum wage policies remain controversial. In the U.S., for example, there is an active debate over the connection between the decline in the real minimum wage and the rise in income inequality over the last decades. Maybe less known, Brazil---among other Latin American countries---has seen a remarkable decline in income inequality since the 1990s. Over the same period, Brazil's real minimum wage more than doubled. This raises the question: Is the minimum wage an effective tool to reduce income inequality?

The main contribution of our paper is to quantify the effects of a large increase in the minimum wage in Brazil from 1996 to 2018 on inequality and employment. By exploiting variation in the effective bindingness of the federal minimum wage across states \citep{Lee1999,Autor2016}, we show that a higher minimum wage is associated with compression throughout most of the wage distribution. At the same time, we find little evidence of negative effects of the minimum wage on employment. To understand these results, we develop an equilibrium model of a frictional labor market subject to a minimum wage, with a particular focus on the role played by heterogeneous firms in mediating such a policy. The minimum wage compresses firm pay differences and impacts wages higher up in the distribution. At the same time, it leads to worker reallocation from less to more productive employers, countering the effect of a (modest) employment decline on aggregate output. We conclude based on our reduced-form and structural analysis that the minimum wage was a key factor behind Brazil's remarkable decline in wage inequality over this period.

Our analysis proceeds in three steps. In the first step, we empirically dissect Brazil's inequality decline and link it to firm heterogeneity and the minimum wage. To this end, we decompose the variance of log wages using a variant of the two-way fixed effects model due to \citet*[][henceforth AKM]{abowdkramarzmargolis1999} estimated within separate time windows. This decomposition allows us to assess whether firms are a key channel through which the minimum wage may reshape the wage distribution. We find that declining firm pay heterogeneity for identical workers, which accounts for 26 percent of the variance of log wages around 1996, explains 43 percent of the reduction in the variance over time. To quantify the fraction of the aggregate decline in inequality that is accounted for by the minimum wage, we exploit cross-sectional variation in the effective bindingness of the federal minimum wage across states. Motivated by the fact that especially lower-tail inequality declined by more in initially lower-income regions, we estimate the effects of the minimum wage throughout the wage distribution building on the seminal econometric framework by \citet{Lee1999} and the recent contribution by \citet{Autor2016}. We find robust evidence of spillover effects of the minimum wage throughout most of the wage distribution and a large negative effect on the standard deviation of wages. At the same time, we find little effect of the minimum wage on employment, formality, and other labor market outcomes.

In the second step, we develop and estimate an equilibrium model of Brazil's labor market subject to a minimum wage to understand these patterns. Our model extends the popular \citet{BurdettMortensen1998} framework to include unobserved worker heterogeneity, minimum wage jobs, and endogenous job creation in a tractable manner. We show that a relatively simple extension of this framework can be operationalized to speak to worker and firm pay differences in the data and to quantify the equilibrium effects of the minimum wage. In our model, workers permanently differ in their ability and value of leisure, as well as their time-varying on-the-job search efficiency and separation rate. They engage in random search in frictional labor markets segmented by worker type. Differentially productive firms operating a linear technology in labor chose what wage to offer and a recruiting intensity in each market. The model allows for a flexible account of worker and firm pay differences, including a mass point in the wage distribution at the minimum wage. We estimate the model via the Simulated Method of Moments (SMM) based on our linked employer-employee data and find that, despite its simplicity, it provides a parsimonious account of salient empirical patterns in Brazil.

In the third and final step, we use the model to quantify the effects of the observed increase in the minimum wage on the distribution of wages, employment, and aggregate output. To this end, we feed the empirical increase in Brazil's minimum wage between 1996 and 2018 into the estimated model. We find that the increased minimum wage reduces the variance of wages by 13 log points, or 45 percent of the empirical decline over this period. A critical factor behind these large effects on inequality is that the rise in the minimum wage induces firms above the new minimum wage to raise pay to maintain their rank in the wage distribution. Indeed, such spillover effects reach all the way to the top of the wage distribution, though the wage gain is a relatively modest six percent at the 50th percentile and two percent at the 75th percentile. We demonstrate that the magnitudes of our estimated effects of the minimum wage on inequality are driven by how binding the minimum wage is, together with the extent of firm productivity dispersion in Brazil. At the same time, we find muted negative effects of the minimum wage on employment and aggregate output due to the heterogeneous effects of the minimum wage across the firm productivity distribution. Lower-productivity firms cut vacancy creation as the minimum wage squeezes their profit margins. The easier recruiting environment in turn induces higher-productivity firms to increase hiring. As a result, the minimum wage primarily reallocates employment from lower- to higher-productivity firms rather than to unemployment.

\paragraph{Related literature.}

This paper contributes to three strands of the literature. First, much research has been devoted to the reduced-form measurement of minimum wage effects on labor market outcomes.%
%
\footnote{See \citet{Card1995} and \citet{NeumarkWascher2008} for comprehensive overviews of this literature.} %
%
A large number of these studies are concerned with the employment effects of the minimum wage \citep[e.g.,][]{CardKrueger1994}. A complementary set of papers assess the distributional consequences of the minimum wage in the U.S. and other high-income countries \citep{Grossman1983, DiNardo1996, Machin2003, Teulings2003, ButcherDickensManning2012, FortinLemieux2015, Brochuetal2018, FirpoFortinLemieux2018, RinzVoorheis2018, CengizDubeLindnerZipperer2019, FortinLemieuxLloyd2021}. In a seminal contribution to this literature, \citet{Lee1999} finds significant effects of the minimum wage in the lower half of the U.S. wage distribution. By extending this methodology and data series, \citet{Autor2016} argue that spillover effects of the minimum wage are indistinguishable from measurement error using household survey data from the U.S. Current Population Survey (CPS). Relative to these papers, we exploit administrative data to quantify the effects of a large increase in the minimum wage in a developing country, Brazil. We find robust evidence of spillovers throughout large parts of the wage distribution, which we link to the relatively greater bindingness of the minimum wage and dispersion in firm pay policies in Brazil.

Second, a separate literature has developed and estimated structural models to assess the impacts of a minimum wage. \Citet{VandenBerg1998}, \citet{Bontemps1999,Bontemps2000}, and \citet{Manning2003} highlight the contribution of firms in imperfectly competitive labor markets toward wage dispersion for identical workers, based on the seminal framework by \citet{BurdettMortensen1998}. A theoretical prediction of this framework is that the minimum wage has spillover effects on higher wages through the equilibrium response of firm pay policies. Perhaps surprisingly, the magnitude of these spillover effects has, before our work, not been quantified using worker-firm linked data. Related research abstracts from firms and instead models match-level heterogeneity to study endogenous contact rates \citep{Flinn2006} and the nature of wage setting \citep{FlinnMullins2018} in the context of minimum wage policies. Relative to these works, we show that a model of multi-worker firms has distinct predictions for the reallocation of workers across heterogeneous employers and changes in firm pay policies in response to a minimum wage. In this sense, our findings connect to recent work on the reallocative effects of minimum wages \citep{AaronsonFrenchSorkinTo2018, BergerHerkenhoffMongey2019, BergerHerkenhoffMongey2021, HarasztosiLindner2019, DustmannLindnerSchonbergUmkehrervomBerge2020, ClemensKahnMeer2021}.%
%
\footnote{Other mechanisms that could give rise to spillover effects include skill assignments with comparative advantage \citep{Teulings1995a}, hierarchical matching \citep{LopesdeMelo2012}, fairness considerations \citep{Card2012a}, educational investment \citep{Barany2016a}, hedonic compensation \citep{Phelan2018}, and the union threat \citep{TaschereauDumouchel2020}.} %
%
Our contribution is to show that a relatively simple extension of the seminal framework by \citet{BurdettMortensen1998} provides a strikingly good description of the Brazilian labor market and is well suited to incorporating minimum wages into the recent literature exploring the role of firms in labor market outcomes \citep{Clemens2021}.%
%
\footnote{See also \citet{Davis1991}, \citet{abowdkramarzmargolis1999}, \citet{Card2013c}, \citet{Barth2016a}, and \citet{SongPriceGuvenenBloomvonWachter2018} for empirical studies of firms in the labor market, and \citet{Postel-Vinay2002}, \citet{DeyFlinn2005}, \citet{Cahuc2006}, \citet{LiseRobin2017}, \citet{bilaletal2019}, \citet{elsbygottfries2019}, \citet{gouinbonenfant2020}, \citet{BilalLhuillier2020}, and \citet{Jarosch2021} for recent structural advances in this area.} %
%
The model also helps to reconcile our finding of large distributional consequences of the minimum wage with its small disemployment effects \citep{Teulings2000} and sheds light on the determinants of the magnitude of these effects \citep{Neumark2017}.%

Third, our paper relates to a literature that aims to understand the evolution of wage inequality in Brazil over the past decades, as summarized by \citet{FirpoPortella2019}. \citet{ABEM2018} document the role of falling firm pay differences in a large inequality decline in Brazil between 1996 and 2012, for which our current paper provides a structural explanation: the rise of the minimum wage. Previous reduced-form work by \citet{Fajnzylber2001}, \citet{NeumarkCunninghamSiga2006}, and \citet{Lemos2009} studies the distributional effects of Brazil's minimum wage over an earlier period before the minimum wage rapidly increased. Subsequent work by \citet{Haanwinckel2020} also quantifies the contribution of the minimum wage toward the decline in wage inequality in Brazil. Although his task-based model differs from ours in several dimensions, his main conclusion is consistent with our results on the inequality-reducing effect of the minimum wage through spillovers higher up in the wage distribution. Like in other developing countries, the informal sector plays an important role in the Brazilian labor market \citep{Ulyssea2018, Ulyssea2020, DixCarneiroGoldbergMeghirUlyssea2021}. While our estimated model accounts for informality in a simple manner, the richer model by \citet{MeghirNarita2015} allows for interactions between formal and informal firms, suggesting that policies like the minimum wage may affect pay and employment in both sectors \citep{Jales2018}.

\paragraph{Outline.}

The remainder of the paper is structured as follows. Section \ref{SECTION: Data} introduces the data and dissects Brazil's inequality decline. Section \ref{SECTION: Empirics} presents reduced-form evidence for the effects of the minimum wage on wages and employment. Section \ref{SECTION: Model} develops a structural equilibrium model of Brazil's labor market subject to a minimum wage. Section \ref{SECTION: Estimation} estimates the model. Section \ref{SECTION: Results} uses the estimated model to quantify the effects of the minimum wage on the distribution of wages and employment. Finally, Section \ref{SECTION: Conclusion} concludes.
