% !TEX root = EIMW2022.tex

\section{The minimum wage and other wage setting institutions in Brazil\label{SECTION: Institutions}}

\subsection{Brazil's minimum wage}

Brazil first adopted a regional minimum wage as part of the decree-law \emph{Decreto-Lei No.\ 2.162} on May 1, 1940, under then-dictator and later-elected-president Get{\'{u}}lio Vargas. In 1984, the regional minimum wages were unified under a federal minimum wage. %
%
Over much of the period we study, the federal minimum wage was the only unconditional wage floor in place. However, since the passage of the labor law \emph{Lei Complementar No.\ 103} on July 14, 2000, states are allowed to institute their own wage floors called \emph{Pisos Salariais Estaduais}. Since then, five out of the 27 states have instituted such state-specific minimum wages \citep{CorseuilFogelHecksher2015,SaltielUrzua2020}.%
%
\footnote{Technically, the Federative Republic of Brazil consists of 26 states and one federal district, the \emph{Distrito Federal}. For simplicity, we henceforth refer to all of Brazil's 27 federative units (i.e., the 26 states and the federal district) as ``states.''} %
%
These five states are located in the relatively high-income southern and southeastern regions of Brazil and comprise Rio de Janeiro and Rio Grande do Sul since 2001, Paran{\'{a}} since 2006, S{\~{a}}o Paulo since 2007, and Santa Catarina since 2010. %
%
Nevertheless, the federal minimum wage (henceforth referred to as the ``minimum wage'') remains the most important wage floor for the majority of the Brazilian population.

Brazil's minimum wage is stated in terms of a floor on monthly nominal earnings, with no provisions for legal subminimum wages or differentiated minimum wages across demographics or economic subdivisions \citep{Lemos2004}. The minimum wage applies to workers with full-time contracts of 44 hours per week, and is adjusted proportionately for part-time workers.%
%
\footnote{Using information on hours in the RAIS data, we find a relatively small share of part-time workers. Special labor contracts allow for parts of the minimum wage to be paid in-kind in the form of accommodation
and food, although in the PNAD data only 0.8 percent of workers report receiving nonmonetary remuneration in 1996.} %


\subsection{Other wage setting institutions}

While the minimum wage serves as an important reference point for wage setting in Brazil, a number of other labor market institutions complement its role. Industry- and occupation-specific trade unions regularly negotiate wage floors for members and other workers with coverage of collective bargaining agreements. During the hyperinflationary period of the early 1990s, wages were commonly expressed as multiples of the minimum wage, though its use as an explicit numeraire has been outlawed and, in practice, is greatly imperfect. Nevertheless, the minimum wage serves as a benchmark for unemployment and retirement benefits. Apart from providing a lower bound on permissible wages, the minimum leaves ample freedom for firms to pay above the minimum wage. In this way, the minimum wage serves as a reference point for wage negotiations.

\subsection{\label{subsec:The-minimum-wage}Evolution of Brazil's minimum wage over time}

Motivated by the remarkable decline in wage inequality in Brazil, we now turn to a
salient change in the labor market over this period: the rise in the
minimum wage.%
%
\footnote{While Brazil enacted other social policies during the mid-2000s, such as a transfer program for needy families (\emph{Bolsa Fam{\'{i}}lia}) launched in 2003, the minimum wage predates many of these policies.} %
%
Brazil's real minimum wage deteriorated under high inflation between 1985 and 1995. A switch in government towards the end of this period ignited a gradual ascent of the wage floor from BRL 500.4 in 1996 to BRL 1,142.3 (both in constant September 2021 BRL) in 2018, which corresponds to a 128.3 percent increase in real terms. Accounting for aggregate productivity growth, this corresponds to a 58.6 log points real productivity-adjusted rise in the minimum wage over 23 years. To put these numbers into context, the minimum wage as a fraction of the median wage increased from around 30.3 percent in 1996 to around 55.6 percent in 2018. Figure \ref{fig:var_minw_short} shows a strong negative comovement between the minimum wage and the standard deviation of log wages between 1996 and 2018, with a time series correlation of $-0.973$.%
%
\footnote{Appendix \ref{appendix: inequality and mw} shows an equally striking comovement between earnings inequality and the minimum wage over the extended period from 1985--2018.} %

\begin{figure}[!htb]
  %
  \centering
  \caption{\label{fig:var_minw_short}Evolution of wage inequality and the real minimum wage}
  %
  \prefigvspace
  %
  \includegraphics[width=0.45\columnwidth]{_figures/fig2.pdf} % _figures/sd_mw_real_short.pdf
  %
  \\
  %
  \postfigvspace
  %
  \begin{minipage}[t]{1\columnwidth}%
    \begin{spacing}{0.75}
      \emph{\scriptsize{}Notes: }{\scriptsize{}Statistics are for males
      of age 18--54. Real minimum wage is the annual mean of the monthly
      time series. The correlation between the two time series is $-0.973$. %
      \emph{\scriptsize{}Source: } RAIS and IPEA, 1996--2018.}
    \end{spacing}
  \end{minipage}
  %
\end{figure}
