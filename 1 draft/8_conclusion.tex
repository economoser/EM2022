% !TEX root = EIMW2022.tex

\section{Conclusion\label{SECTION: Conclusion}}

There remains a great debate over the potential for labor market institutions to affect wage inequality. In this paper, we study a large increase in the minimum wage in Brazil using rich administrative and household survey data together with an equilibrium model to shed new light on this debate. Both our reduced-form analysis, based on variation in the bindingness of the minimum wage across Brazilian states, and our estimated structural model indicate significant scope for the minimum wage to compress the distribution of wages, while having only modest disemployment effects. Through the lens of our equilibrium model and consistent with our reduced-form findings, these results are due to far-reaching spillover effects of the minimum wage on firm pay policies as well as worker reallocation across firms.

Our study points to several fruitful avenues for future research. First, while our structural model incorporates a rather simple view of informality, it would be interesting to quantify spillovers of the minimum wage in Brazil's formal sector to jobs in the informal sector, which is not directly constrained by the policy---what \citet{NeriMoura2006} call the lighthouse effect. Second, given our findings on the prominent role played by firms in the labor market, it is worth revisiting the effects of other labor market policies and institutions---including unions, unemployment benefits, and noncompete agreements---on the distribution of pay and employment in other settings. While such labor market institutions and policies may only affect a small share of workers directly, they may lead to sizable equilibrium effects of the kind we find in Brazil. Finally, our work stops short of an analysis of optimal minimum wage policies in a frictional environment, though our results will be an important ingredient for any such venture.

