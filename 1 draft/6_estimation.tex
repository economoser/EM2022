% !TEX root = EIMW2022.tex

\section{Estimation\label{SECTION: Estimation}}

We estimate the model by targeting empirical moments from the preperiod 1994--1998. The goal is to use the estimated model to quantify the equilibrium effects of the observed increase in the minimum wage.

\subsection{Estimation strategy}

To accommodate unobserved heterogeneity among workers and firms, our model features a continuum of parameters. To reduce the dimensionality of the estimation problem, we make some simplifying assumptions. We first discretize both worker ability $a$ and firm productivity $p$. We then parameterize how worker heterogeneity varies across ability levels $a$ and how firm productivity $p$ is distributed. Subsequently, we proceed in three steps. First, we preset three parameters based on standard values in the literature. Second, we directly infer three parameters, which the model maps one-to-one to three empirical moments. Third, we estimate 12 remaining parameters using the SMM via indirect inference.


\paragraph{Preset parameters.}

We adopt a monthly frequency and set the discount rate to the equivalent of an annual real interest rate of five percent. %
%
We normalize matching efficiency to $\chi = 1$, since without data on vacancies it is not separately identified from the intercept in the vacancy cost function. Based on standard values in the literature \citep{petrongolopissarides2001}, we set the elasticity of matches with respect to vacancies to $\alpha=0.5$, which is at the upper end of the range considered by \citet{MeghirNarita2015}. For robustness, we consider alternative values for $\alpha$ and other key parameters in Appendix \ref{app_subsec:effects_dependence_on_parameters}.


\paragraph{Directly inferred parameters.}

The mass of firms, $M$, can be directly chosen to target a mean firm size of 11.8 workers in RAIS. Under the assumption that the separation rate of workers with zero on-the-job search efficiency, $\delta(a,0)$, is constant across ability levels, we can equate this parameter to the empirical separation rate of workers earning the minimum wage, which equals 6.5 percent per month. We assume that the job finding rate $p(a,s)=\lambda$ is independent of worker ability $a$ and relative on-the-job search efficiency $s$. We set the auxiliary parameter $\lambda$ to target a monthly nonemployment-to-employment (NE) rate of 4.4 percent.%
%
\footnote{Brazil's NE rate is low in an international comparison \citep{engbom2021}, likely because we include informal workers in our definition of nonemployment. This is not a prime concern for us, however, because the key factor affecting firm wages is how fast workers move up and fall off the job ladder, which relates to job-to-job (EE) and employment-to-nonemployment (EN) rates. In contrast, the NE rate impacts the economy primarily through the stock of nonemployed, as we confirm in robustness exercises in Section \ref{subsec:comparative_statics}.} %
%
Of course, $\lambda$ is an equilibrium outcome, but we can treat it as an auxiliary parameter since the cost of creating jobs, $c(a,s)$, can be chosen to rationalize any positive value of $\lambda$ in equilibrium. Hence, we pin down the structural parameters $c(a,s)$ flexibly in each market such that the equilibrium job finding rate is $\lambda$.


\paragraph{Internally estimated parameters.}

We estimate the remaining model parameters by the SMM via indirect inference. Specifically, we choose the parameter vector $\mathbf{p}^{\ast} \in \mathscr{P}$ that minimizes the sum of weighted squared percentage deviations between a set of moments in the model and in the data:
%
\begin{eqnarray*}
  \mathbf{p}^{\ast} &=& \operatorname{arg \ min}\displaylimits_{\mathbf{p}} \sum_{p\in\mathscr{P}} \sum_{m\in \mathscr{M}(p)} w_m \left(\frac{m^{\text{model}} - m^{\text{data}}}{m^{\text{data}}}\right)^2
\end{eqnarray*}
%
While all parameters are jointly determined, we assign to each parameter $p$ a set of moments $\mathscr{M}(p)$ that are particularly informative for $p$ as we compare the model-based moments $m^{\text{model}}$ against their data equivalent $m^{\text{data}}$ with weight $w_{m}$. We discuss our choice of moments and weights in greater detail below.

To further simplify the problem, we impose some flexible parametric restrictions based on inspection of the data vis-{\`{a}}-vis the model output. We assume that log worker ability is distributed according to a double exponential distribution with mean $\mu$ and shape parameter $\sigma$. Firm productivity is Pareto distributed with shape parameter $\zeta$ and a scale parameter normalized to one.

We restrict search efficiency to fluctuate between $s = 0$ and a positive value $s(a)>0$ that depends on ability $a$. In equilibrium, firms offer the reservation wage $r(a,s)a$ to workers with $s=0$ who do not search on the job \citep{Diamond1971}. If the minimum wage binds with $w^{min} = r(a,s)a$ for a positive measure of workers, then our model produces a spike at the minimum wage in the wage distribution. We assume that an employed worker with search efficiency $s(a)>0$ who becomes unemployed transitions to $s(a)=0$ with probability $\pi$ and retains $s(a)$ with complementary probability $1-\pi$.%
%
\footnote{While the probability of transiting to $s=0$ upon separating to unemployment is independent of $a$, our model features a lower incidence of minimum wage jobs among higher-ability workers since they are less likely to separate to unemployment.} %
%
A worker with $s(a)=0$ who becomes unemployed transitions to $s(a) > 0$ with probability $1$. That is,
%
\begin{eqnarray*}
  \pi(x|a,s(a)>0) &=& \left\{ \begin{array}{cc} 1-\pi & \text{if } x = s(a) \\ \pi & \text{if } x = 0 \\ 0 & \text{otherwise} \end{array} \right. \hspace{.5in} \pi(x|a,0) \ \ = \ \ \left\{ \begin{array}{cc} 1 & \text{if } x = s(a) \\ 0 & \text{otherwise} \end{array} \right.
\end{eqnarray*}
%
For a worker with search efficiency $s(a)>0$, the exogenous separation rate is assumed to be an affine transformation of a worker's ability rank, $\delta(a,s)=\delta_0(1 +\delta_1 \Psi(a))$. The relative on-the-job search efficiency among workers in the regular state is $s(a)=\phi_0(1+\phi_1 (\exp(\Psi(a))-1))$. These parametric forms are guided by what appears to fit the data well.

Next, we posit a reduced-form relationship for the reservation wage among workers with positive search efficiency given by $ar(a,s>0)=r_0 + r_1(a-\underline{a})$. The reservation wage is an endogenous outcome, but the flow value of leisure $b(a)$ is a free parameter, allowing us to treat $r(a,s>0)$---or, in this case, $r_0$ and $r_1$---as auxiliary parameters to be estimated. We then choose $b(a)$ so as to reproduce the estimated reservation piece rate $r(a,s)$ as an equilibrium outcome.%
%
\footnote{We verify that all worker types $(a,s=0)$ prefer being employed at the minimum wage over unemployment under our estimated parameter values. Note that in markets where the minimum wage is binding, the minimum wage provides an upper bound on the latent reservation wage. Since the impact of a simulated minimum wage increase---unlike in the case of a decrease---is invariant to the level of the flow value of leisure, $b(a)$, in markets where the minimum wage is initially binding, we assume that $b(a)$ equated to the value of unemployment in those markets.} %
%
This approach allows us to solve the model using the system of differential equations in \eqref{eq: differential equation system} without reference to workers' value functions \eqref{eq: value of unemployment}--\eqref{eq: value of employment}, which leads to a great reduction in computational time.


\paragraph{Model solution, simulation, and estimation.}

We solve the model in continuous time over 50 grid points for ability and 500 grid points for productivity. We then simulate the model at monthly frequency over a period of five years for a large number of workers, starting from the ergodic distribution.
%
To match the empirical residual wage dispersion conditional on worker and firm heterogeneity, we assume that the logarithm of measured wages, $\log \tilde{w}$, equals the sum of the logarithm of the true wage, $\log w$, and measurement error, $\kappa$, so $\log \tilde{w} = \log w + \kappa$. We let $\kappa \sim \mathcal{N}(0,\varepsilon)$ with variance $\varepsilon$ and values drawn independently and identically distributed across worker-firm matches. Motivated by the empirical existence of a (relatively small) spike in the wage distribution at the minimum wage, we assume that measurement error is identically zero for minimum wage jobs. One interpretation of this is that employers offering exactly the minimum wage are well aware of its statutory level and the penalties for violations, which induces them to make accurate reports. We construct monthly and annual data sets based on model simulations, using the same sample selection criteria and variable construction as in the data.

These assumptions leave us with a vector $\mathbf{p}$ of 12 parameters to be estimated using the SMM via indirect inference:%
%
\footnote{Recall that our directly inferred estimate of $\lambda$ is associated with an implied vacancy cost scalar $c(a,s)$ for each market $(a,s)$ and each value of $r(a,s>0)$ corresponding to our estimates of $(r_0, r_1)$ is associated with an implied flow value of leisure $b(a)$.} %
%
\begin{eqnarray*}
  \mathbf{p} &=& \left\{ \mu , \sigma , \zeta , \eta , \varepsilon , \delta_0 , \delta_1 , \phi_0 , \phi_1 , \pi , r_{0} , r_{1} \right\}
\end{eqnarray*}
%
While all parameters are jointly determined, it is useful to provide a heuristic discussion of what data moments are particularly informative for each parameter. We verify this intuition in Appendix \ref{app_subsec:identification}. The scale of the ability distribution, $\mu$, is informed by the log ratio of the median to minimum wage. Greater $\mu$ means that wage distribution is further removed from the wage floor. This moment plays a key role in our analysis and we assign it a weight of $w_{m}=5$. For the ability shape parameter, $\sigma$, we target log wage percentile ratios relative to the median in increments of five (i.e., P5-50, P10-50, \ldots, P95-P50). We assign each of the 18 percentile ratios a weight of $w_{m}=1$.

For the remaining parameters, we connect our equilibrium model to reduced-form estimates from the AKM wage equation in Section \ref{subsec:Motivating-facts}. The AKM wage equation does not have a structural interpretation in our framework. Nevertheless, Appendix \ref{app_subsec:identification} shows that this indirect inference approach disciplines the distributions of unobserved worker and firm heterogeneity in our model vis-{\`{a}}-vis the data.%
%
\footnote{For this indirect inference estimation step, both in our model and in the data, we drop minimum wage workers, do not apply a KSS bias correction, and do not include additional controls. Importantly, we treat the model and the model identically.} %
%

The shape of the Pareto distribution for firm productivity, $\zeta$, is informed by the standard deviation of AKM firm fixed effects. Lower values of $\zeta$ are associated with greater dispersion in productivity and firm pay. We match the curvature of the vacancy cost, $\eta$, to the share of employment at firms with 50 or more workers. For lower values of $\eta$, it is cheaper for firms to scale up vacancies, which results in more productive firms growing relatively larger. Both moments are assigned a weight of $w_{m}=1$.

The variance of measurement error $w_{m}=1$ intuitively maps into the variance of residuals in the AKM wage equation. We assign this moment a weight of $w_{m} = 1$.

For the separation rate's intercept, $\delta_0$, and slope, $\delta_1$, we target the EN rate by AKM worker fixed effect deciles. The intercept $\delta_0$ steers the average EN rate, while the slope in ability, $\delta_1$, steers heterogeneity in EN rates across AKM worker fixed effects. Moments for each of the then AKM worker fixed effect deciles receives a weight of $w_{m}=1/10$, which results in a unit cumulative weight.

The intercept, $\phi_0$, and slope, $\phi_1$, of the relative on-the-job search intensity, $s(a)$ maps into the EE rate by AKM worker fixed effect decile. Again, each of these 10 moments receives a weight of $w_{m}=1/10$.

The probability $\pi(0|a,s)$ that a displaced worker transitions from $s>0$ to $s=0$ maps into the spike at the minimum wage in the wage distribution. This moment also receives a weight of $w_{m}=1$.

For the auxiliary parameters governing reservation wages, $r_0$ and $r_1$, we target the P5 of log wages by AKM worker fixed effect decile. Intuitively, $r_0$ guides the minimum wage bindingness for all markets, while $r_1$ guides the relative bindingness across AKM worker fixed effect deciles. Again, each of these 10 moments receives a weight of $w_{m}=1/10$, which results in a unit cumulative weight.




\subsection{Parameter estimates and model fit\label{subsec:parameter_estimates_fit}}

Table \ref{table: estimates} presents the three preset, three directly inferred, and 12 internally estimated parameter values along with their targeted moments. A few comments are in order, beginning with the set of parameters related to the wage distribution. The model closely replicates the empirical median-to-minimum log wage ratio (related to $\mu=0.960$) and the general shape of the log wage distribution (related to $\sigma=0.258$) shown in Figure \ref{figure: model fit} and to be discussed shortly. A tail index of the firm productivity distribution of $\zeta = 3.503$ allows the model to match well the variance of AKM firm fixed effects. To match the share of workers employed at firms with at least 50 employees, the model requires a relatively low curvature of the vacancy cost, $1 + \eta = 1.467$.%
%
\footnote{Because $\eta$ also governs the elasticity of vacancy creation with respect to firm profitability, a low value of $\eta$ implies that firms' employment responds relatively flexibly to the minimum wage---see Figure \ref{figure: identification 1}\subref{figure: identification 1d} in Appendix \ref{app_subsec:identification} for details.} %
%
Finally, most of the AKM residual variance is accounted for by measurement error, $\varepsilon=0.215$, as opposed to violations of log additivity of the wage equation.

We now turn to a set of parameters related to employment transitions. In the RAIS data, around four percent of workers leave formal employment in the subsequent month, which is close to the EN rate in the U.S. Note that this number includes workers who leave for informal employment not recorded in RAIS. Furthermore, the data show a steep negative gradient between the EN rate and AKM person fixed effect deciles. Together, these empirical moments lead us to estimate $\delta_{0} = 0.074$ and $\delta_{1} = -0.815$---see Figure \ref{figure: model fit 2}\subref{figure: model fit 2a} in Appendix \ref{app_subsec:model_fit} for details. Next, an average of 1.8 percent of workers make an EE transition each month, which is again close to the corresponding number in the U.S.%
%
\footnote{Using survey data from the PME, \citet{MeghirNarita2015} report a quarterly EE rate of 1.58 percent and 2.49 percent, respectively, in the Brazilian metropolitan regions of S{\~{a}}o Paulo and Salvador. There are several differences between the way we estimate EE transitions for our purposes compared to \citet{MeghirNarita2015}. Our estimates are based on a different data set, RAIS, which has wider geographic coverage. RAIS, unlike PME, also records the exact employment start and end dates, mitigating concerns about time aggregation bias \citep{Shimer2012}. Compared to survey data like PME, reporting issues are likely also a lesser concern in administrative data like RAIS. Finally, regarding right censoring, the RAIS data allow us to estimate transition rates over a longer panel of sixty months, compared to the four-month panel in PME. See \citet{EngbomGonzagaMoserOlivieri2021} for a detailed comparison between the PME and RAIS datasets.} %
%
Because the EE rate is high relative to the NE rate in Brazil, we infer a high average relative search efficiency, $s(a)$. This does not mean that Brazilian labor markets are highly efficient but merely that EE transitions are not as rare as NE transition rates may suggest.%
%
The resulting parameter estimates $\phi_0=0.436$ and $\phi_1=1.055$ match the empirical EE transition rates shown in Figure \ref{figure: model fit 2}\subref{figure: model fit 2b} of Appendix \ref{app_subsec:model_fit}. The estimated value of the transition rate to minimum wage jobs, $\pi=0.019$, leads our model to generate a realistic spike in the wage distribution at the minimum wage.

The two remaining parameters relate to workers' outside option value. The parameters $r_0 = -0.078$ and $r_1 = 1.127$ capture the empirical feature of the P5 of log wages rising steeply across AKM person fixed effect deciles---see Panel \subref{figure: model fit 2d} of Figure \ref{figure: model fit 2} in Appendix \ref{app_subsec:model_fit}.


\begin{table}[!htb]
  %
  \centering
  \caption{Parameter estimates\label{table: estimates}}
  %
  \resizebox{\textwidth}{!}{%
    \begin{tabular}{l l c l c c} \hline \hline \addlinespace[1ex] 
 \multicolumn{2}{l}{Parameter} & Estimate & Targeted moment & Data & Model \\ 
\hline \addlinespace[1.5ex] 
& \multicolumn{5}{c}{\textit{Panel A. Pre-determined parameters}} \\ \cline{2-6} \addlinespace[1ex] 
$\rho$ & Discount rate & 0.004 & 4\% annual real interest rate \\ 
$\chi$ & Matching efficiency & 1.000 & Normalization \\ 
$\alpha$ & Elasticity of matches w.r.t. vacancies & 0.500 & \citet{petrongolopissarides2001} \\ 
\addlinespace[1.5ex] 
& \multicolumn{5}{c}{\textit{Panel B. Structural and auxiliary parameters calibrated offline}} \\ \cline{2-6} \addlinespace[1ex] 
$M$ & Mass of firms & 0.069 & Average firm size & 11.787 & 13.085 \\ 
$\delta(a,0)$ & Separation rate of those with $s=0$ & 0.065 & EN rate from MW jobs & 0.064 & 0.065 \\ 
$\lambda$ & Job finding rate & 0.044 & NE rate & 0.044 & 0.044 \\ 
\addlinespace[1.5ex] 
& \multicolumn{5}{c}{\textit{Panel C. Internally estimated structural parameters}} \\ \cline{2-6} \addlinespace[1ex] 
$\mu$ & Mean of worker ability & 0.960 & Median to minium wage & 1.224 & 1.192 \\ 
$\sigma$ & Shape of worker ability & 0.258 & Percentiles of wage distribution & \multicolumn{2}{c}{See figure \ref{figure: model fit}} \\ 
$\zeta$ & Shape of productivity distribution & 3.503 & Variance of AKM firm FEs & 0.217 & 0.195 \\ 
$\eta$ & Curvature of vacancy cost & 0.467 & Employment share of firms with 50+ empl. & 0.588 & 0.583 \\ 
$\epsilon$ & Variance of noise & 0.215 & Variance of AKM residual & 0.032 & 0.035 \\ 
$\delta_0$ & Separation rate, intercept & 0.074 & EN rate & \multicolumn{2}{c}{See figure \ref{figure: model fit 2}} \\ 
$\delta_1$ & Separation rate, slope & -0.815 & EN rate & \multicolumn{2}{c}{See figure \ref{figure: model fit 2}} \\ 
$\phi_0$ & Relative search intensity, intercept & 0.436 & JJ rate & \multicolumn{2}{c}{See figure \ref{figure: model fit 2}} \\ 
$\phi_1$ & Relative search intensity, slope & 1.055 & JJ rate & \multicolumn{2}{c}{See figure \ref{figure: model fit 2}} \\ 
$\pi$ & Transition rate to MW & 0.019 & Share of employed earning the MW & 0.012 & 0.011 \\ 
\addlinespace[1.5ex] 
& \multicolumn{5}{c}{\textit{Panel D. Internally estimated auxiliary parameters}} \\ \cline{2-6} \addlinespace[1ex] 
$r_0$ & Reservation wage, intercept & -0.078 & 5th wage percentile & \multicolumn{2}{c}{See figure \ref{figure: model fit 2}} \\ 
$r_1$ & Reservation wage, slope & 1.127 & 5th wage percentile & \multicolumn{2}{c}{See figure \ref{figure: model fit 2}} \\ 
\addlinespace[.5ex] \hline 
\end{tabular} % \begin{tabular}{l l c l c c} \hline \hline \addlinespace[1ex] 
 \multicolumn{2}{l}{Parameter} & Estimate & Targeted moment & Data & Model \\ 
\hline \addlinespace[1.5ex] 
& \multicolumn{5}{c}{\textit{Panel A. Pre-determined parameters}} \\ \cline{2-6} \addlinespace[1ex] 
$\rho$ & Discount rate & 0.004 & 4\% annual real interest rate \\ 
$\chi$ & Matching efficiency & 1.000 & Normalization \\ 
$\alpha$ & Elasticity of matches w.r.t. vacancies & 0.500 & \citet{petrongolopissarides2001} \\ 
\addlinespace[1.5ex] 
& \multicolumn{5}{c}{\textit{Panel B. Structural and auxiliary parameters calibrated offline}} \\ \cline{2-6} \addlinespace[1ex] 
$M$ & Mass of firms & 0.069 & Average firm size & 11.787 & 13.085 \\ 
$\delta(a,0)$ & Separation rate of those with $s=0$ & 0.065 & EN rate from MW jobs & 0.064 & 0.065 \\ 
$\lambda$ & Job finding rate & 0.044 & NE rate & 0.044 & 0.044 \\ 
\addlinespace[1.5ex] 
& \multicolumn{5}{c}{\textit{Panel C. Internally estimated structural parameters}} \\ \cline{2-6} \addlinespace[1ex] 
$\mu$ & Mean of worker ability & 0.960 & Median to minium wage & 1.224 & 1.192 \\ 
$\sigma$ & Shape of worker ability & 0.258 & Percentiles of wage distribution & \multicolumn{2}{c}{See figure \ref{figure: model fit}} \\ 
$\zeta$ & Shape of productivity distribution & 3.503 & Variance of AKM firm FEs & 0.217 & 0.195 \\ 
$\eta$ & Curvature of vacancy cost & 0.467 & Employment share of firms with 50+ empl. & 0.588 & 0.583 \\ 
$\epsilon$ & Variance of noise & 0.215 & Variance of AKM residual & 0.032 & 0.035 \\ 
$\delta_0$ & Separation rate, intercept & 0.074 & EN rate & \multicolumn{2}{c}{See figure \ref{figure: model fit 2}} \\ 
$\delta_1$ & Separation rate, slope & -0.815 & EN rate & \multicolumn{2}{c}{See figure \ref{figure: model fit 2}} \\ 
$\phi_0$ & Relative search intensity, intercept & 0.436 & JJ rate & \multicolumn{2}{c}{See figure \ref{figure: model fit 2}} \\ 
$\phi_1$ & Relative search intensity, slope & 1.055 & JJ rate & \multicolumn{2}{c}{See figure \ref{figure: model fit 2}} \\ 
$\pi$ & Transition rate to MW & 0.019 & Share of employed earning the MW & 0.012 & 0.011 \\ 
\addlinespace[1.5ex] 
& \multicolumn{5}{c}{\textit{Panel D. Internally estimated auxiliary parameters}} \\ \cline{2-6} \addlinespace[1ex] 
$r_0$ & Reservation wage, intercept & -0.078 & 5th wage percentile & \multicolumn{2}{c}{See figure \ref{figure: model fit 2}} \\ 
$r_1$ & Reservation wage, slope & 1.127 & 5th wage percentile & \multicolumn{2}{c}{See figure \ref{figure: model fit 2}} \\ 
\addlinespace[.5ex] \hline 
\end{tabular}
  }
  %
  \\
  %
  \posttabvspace
  %
  \begin{minipage}[t]{1\columnwidth}%
    \begin{spacing}{0.75}
      \emph{\scriptsize{}Notes: }{\scriptsize{}Parameter estimates are expressed at a monthly frequency, when applicable. %
      \emph{\scriptsize{}Source: } Model and RAIS, 1994--1998.}
    \end{spacing}
  \end{minipage}
  %
\end{table}


We now discuss the mapping between the estimated auxiliary parameters ($\lambda$, $r_0$, $r_1$) and the corresponding structural parameters of the model. Panel \subref{figure: estimates a} of Figure \ref{figure: estimates} plots the implied vacancy cost scalars $c(a,s)$ across markets $(a,s)$. The implied per-ability-unit vacancy cost is nonmonotonic, initiall decreasing in ability, and subsequently increasing. %
Because the overall recruiting cost for workers of type $(a,s)$ equals $ac(a,s)$, the overall recruiting cost turns out to be relatively flat among low ability levels and then sharply increasing toward higher ability levels. Conditional on worker ability $a$, recruiting costs are uniformly higher in the markets with $s=0$.

Panel \subref{figure: estimates b} of Figure \ref{figure: estimates} shows the flow value of leisure $b(a)a$ across ability types $a$, which is first flat and then upward-sloping, consistent with the idea that higher-ability workers are also better at home production or at work in the informal sector. Panel \subref{figure: estimates c} shows for each ability type the flow value as a fraction of mean wages, $\overline{w}(a)$, which varies from around 80 percent among low ability levels to around 40 percent at medium and high ability levels. Appendix \ref{app_subsec:mean_to_min} shows that these estimates give rise to a model-implied mean-to-min wage ratio \citep{Hornstein2011} of between 1.3 at low ability levels and 3.0 at the top. Thus, the estimated model suffers less from the critique raised by \citet{Hornstein2011} that many search models require unrealistically low (or indeed negative) flow values of leisure to generate realistic levels of frictional wage dispersion. This result obtains for two reasons. First, the high relative on-the-job search efficiency means that job acceptance out of unemployment forgoes a lesser option value. Second, a large share of the variance of wages in our linked employer-employee data is due to unobserved worker heterogeneity, corresponding to ability differences in our model. Conventional survey data would attribute this variation partly to residual, or frictional, wage dispersion---see Appendix \ref{app_subsec:unobserved_heterogeneity} for details.


\begin{figure}[!htb]
  %
  \centering
  \caption{Estimated vacancy costs and flow values of leisure\label{figure: estimates}}
  %
  \prefigvspace
  %
  \subfloat[Scalar of vacancy cost\label{figure: estimates a}]{\includegraphics[trim={0in 0in 0in .3in},clip,width=.33\columnwidth]{_figures/fig5A.png}} % _figures/Model_cost.png
  \subfloat[Flow value of leisure\label{figure: estimates b}]{\includegraphics[trim={0in 0in 0in .3in},clip,width=.33\columnwidth]{_figures/fig5B.png}} % _figures/Model_b.png
  \subfloat[Flow value of leisure/avg. wage\label{figure: estimates c}]{\includegraphics[trim={0in 0in 0in .3in},clip,width=.33\columnwidth]{_figures/fig5C.png}} % _figures/Model_bw.png
  %
  \\
  %
  \postfigvspace
  %
  \begin{minipage}[t]{1\columnwidth}%
    \begin{spacing}{0.75}
      \emph{\scriptsize{}Notes: }{\scriptsize{}Parameter estimates are expressed at a monthly frequency, when applicable. Panel \subref{figure: estimates a} shows the scalar $c(a,s)$ of the vacancy cost function $ac(a,s)v^{1+\eta}/(1+\eta)$. Panel \subref{figure: estimates b} shows the flow value of leisure $b(a)a$ that workers of ability $a$ receive when not formally employed. Panel \subref{figure: estimates c} shows the flow value of leisure, $b(a)$, relative to the ability specific average wage, $\overline{w}(a)=\int_{s}\int_z w(z|a,s)dG(z|a,s)d\Phi(s|a)$, where $\Phi(s|a)$ is the conditional distribution of workers of ability $a$ over search efficiency, $s$. %
      \emph{\scriptsize{}Source: } Model.}
    \end{spacing}
  \end{minipage}
  %
\end{figure}


We now turn to an important dimension of our model's performance, namely its fit vis-{\`{a}}-vis the empirical wage distribution. Figure \ref{figure: model fit} compares the distribution of log wages in the data and the model. Overall, the model-generated wage distribution matches several salient features of the empirical wage distribution. These include its mode, dispersion, skewness, a spike at the minimum wage, and a mass below the minimum wage. At the same time, the model fit is less than perfect. For example, the model underpredicts the mass in the far right tail of the wage distribution. It also overpredicts the number of workers below the minimum wage. While the model matches well the spike exactly at the minimum wage---see Table \ref{table: estimates} above---it slightly understates the number of workers earning just above the minimum wage.%
%
\footnote{Appendix \ref{app_subsec:effects_dependence_on_parameters} shows that our results are robust to varying parameters to better fit these features of the data in isolation.} %
%
We postulate that more flexible parametric forms or a richer wage setting mechanisms such as that in \citet{FlinnMullins2018} would help match these features.%
%
\footnote{\citet{FlinnMullins2018} show that the presence of wage bargaining, in addition to wage posting, can change the predicted spillover effects of the minimum wage. We think that our wage posting model provides a good approximation for our problem at hand for two reasons. First, low-skill workers have been shown to be less likely to bargain over wage in the U.S. \citep{Hall2012}. Given that the average skill level is significantly lower in Brazil, it is reasonable to expect wage bargaining to be relatively rare for most of Brazil's labor force. Second, we show in Section \ref{subsec:spillovers_model_data} that our model predicts minimum wage spillovers in line with our reduced-form estimates for most of the wage distribution, suggesting that an added degree of freedom from integrating a parameter that guides the trade-off between posting and bargaining would marginally improve the model's predictive power in our context.} %
%
We note, however, that such extensions would come at a significant increase in computational time, which is already substantial.%
%
\footnote{Appendix \ref{app_subsec:model_fit} presents further details of the model's fit to the data. Appendix \ref{app_subsec:identification} contains additional estimation diagnostics. We find that most of the parameters are well identified. The only exception is the intercept in the reservation wage, $r_0$, which Appendix \ref{app_subsec:effects_dependence_on_parameters} shows has a negligible impact on the predicted effects of the minimum wage.} %
%

\begin{figure}[!htb]
  %
  \centering
  \caption{Distribution of wages in estimation period, model vs. data \label{figure: model fit}}
  %
  \prefigvspace
  %
  \includegraphics[trim={0in .1in 0in .3in},clip,width=.45\columnwidth]{_figures/fig6.png} % _figures/WageDistribution_1994_1998.png
  %
  \\
  %
  \postfigvspace
  %
  \begin{minipage}[t]{1\columnwidth}%
    \begin{spacing}{0.75}
      \emph{\scriptsize{}Notes: }{\scriptsize{}Log monthly earnings, expressed as a multiple of the current minimum wage, and constructed as the sum of earnings from a given employer over the five year sample period divided by the sum of months worked for that employer over the five year sample period. Sample selection and variable construction criteria for the model are chose to match those of the data. %
      \emph{\scriptsize{}Source: } Model and RAIS, 1994--1998.}
    \end{spacing}
  \end{minipage}
  %
\end{figure}




\subsection{Worker-firm sorting and firm pay}

It will be instructive to lay out the mechanics of the estimated model with regards to worker-firm sorting and firm pay. Regarding sorting, panel \subref{figure: model mechanics a} of Figure \ref{figure: model mechanics}  shows that higher-ability workers work at more productive firms, which rationalizes the positive correlation between AKM worker and firm fixed effects we documented in Section \ref{subsec:Motivating-facts}. This is the case even absent log-complementarities in the production technology, since we estimate that higher-skill workers are more efficient at climbing the job ladder. However, a binding minimum wage causes the assortative matching to be negative near the bottom of the ability distribution because it renders matches between low-skill workers and low-productivity firms unviable. Figure \ref{figure: model validation} in Appendix \ref{app_subsec:model_validation} provides reduced-form evidence consistent with this prediction.

Panel \subref{figure: model mechanics b} of Figure \ref{figure: model mechanics} shows piece rates across firm productivity levels for a group of workers most affected by the minimum wage---specifically, the first percentile of worker ability. More productive firms pay identical workers more to grow larger. At the same time, pay increases less than one-for-one with productivity. Consequently, higher productivity firms have a lower labor share \citep{gouinbonenfant2020}.


\begin{figure}[!htb]
  %
  \centering
  \caption{Model mechanics \label{figure: model mechanics}}
  %
  \prefigvspace
  %
  \subfloat[Mean firm productivity by worker ability\label{figure: model mechanics a}]{\includegraphics[trim={.1in .12in 0in .3in},clip,width=.40\columnwidth]{_figures/fig7A.png}}\hspace{.4in} % _figures/ModelMechanics_sorting.png
  \subfloat[Piece rate by firm productivity\label{figure: model mechanics b}]{\includegraphics[trim={.1in .12in 0in .3in},clip,width=.40\columnwidth]{_figures/fig7B.png}} % _figures/ModelMechanics_gradient.png
  %
  \\
  %
  \postfigvspace
  %
  \begin{minipage}[t]{1\columnwidth}%
    \begin{spacing}{0.75}
      \emph{\scriptsize{}Notes: }{\scriptsize{}Panel \subref{figure: model mechanics a} shows the average log firm productivity by worker ability, $\int_{\underline{z}}^{\overline{z}} \log z dG(z|a,s)$, in $s(a)>0$ market by worker ability. Panel \subref{figure: model mechanics b} shows log piece rates, $\log w(z|a,s)$, offered by firms to the first percentile of the worker ability distribution in market for $s(a)>0$ workers. %
      \emph{\scriptsize{}Source: }\scriptsize{}Model.}
    \end{spacing}
  \end{minipage}
  %
\end{figure}
