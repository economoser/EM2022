% !TEX root = EIMW2022.tex

\section{Results Appendix\label{APPENDIX: Results}}

This section provides additional details on the simulated impact of the minimum wage, building on the material presented in Section \ref{SECTION: Results}, including subsections on %
%
the impact of the minimum wage on inequality across space in the model and data for alternative specifications (Appendix \ref{app_subsec:comparing_spillovers_model_data}),%
%
a model-based AKM wage decomposition (Appendix \ref{app_subsec:akm_decomposition}), %
%
the effect of the minimum wage on young workers only (Appendix \ref{app_subsec:young_only}), %
%
the impact of the minimum wage on sorting (Appendix \ref{app_subsec:impact_sorting}), %
%
further results on heterogeneity in effects on disemployment and firm size (Appendix \ref{app_subsec:employment_effects_het_and_fsize}), %
%
and the dependence of minimum wage effects on model parameters (Appendix \ref{app_subsec:effects_dependence_on_parameters}).



\clearpage
\subsection{Comparing estimated spillover effects between model and data\label{app_subsec:comparing_spillovers_model_data}}

Figure \ref{fig: Lee data model vs data alternative specs} shows estimates of the marginal effects from equation \eqref{eq:Lee_marginal_effect} based on the regression framework in equation \eqref{eq:Lee} using an IV strategy. The results are broadly similar with those from our baseline specification in Figure \ref{fig: Lee data model vs data} of the main text.

\begin{figure}[!htb]
  %
  \centering
  \caption{Model vs. data: Estimated minimum wage effects throughout the wage distribution, IV\label{fig: Lee data model vs data alternative specs}}
  %
  \prefigvspace
  %
  \hspace*{\fill}%
  \csubfloat[Relative to P50\label{fig: Lee data model vs data alternative specs A}]{%
   \includegraphics[width=0.49\columnwidth]{_figures/figE1A.pdf}% _figures/comp_state_trend_1_iv_p50_se1_1996_2018.pdf
    %
  }\centerhfill[\qquad\qquad\qquad\qquad\qquad]
  \csubfloat[Relative to P90\label{fig: Lee data model vs data alternative specs B}]{%
   \includegraphics[width=0.49\columnwidth]{_figures/figE1B.pdf}% _figures/comp_state_trend_1_iv_p90_se1_1996_2018.pdf
    %
  }\hspace*{\fill}
  %
  \\
  %
  \postfigvspace
  %
  \begin{minipage}[t]{1\columnwidth}%
    \begin{spacing}{0.75}
      \emph{\scriptsize{}Notes:}{\scriptsize{}Figure plots estimates of the marginal effects from equation \eqref{eq:Lee_marginal_effect} based on the regression framework in equation \eqref{eq:Lee} estimated across Brazil's 27 states. Results from four separate estimates are shown, namely the combination of two base percentiles---P50 (panel \subref{fig: Lee data model vs data alternative specs A}) and P90 (panel \ref{fig: Lee data model vs data alternative specs B})---and two sources---the RAIS data (black circles and solid lines) and model-simulated data (magenta crosses and dashed lines). All four sets of estimates use a specification that includes state fixed effects in addition to state-specific linear time trends, estimated using an IV strategy. The IV strategy instruments the Kaitz-$p$ index and its square using an instrument set that consists of the log real statutory minimum wage, its square, and the log real statutory minimum wage interacted with the mean of the log real median wage for the region over the full sample period. Within each panel, the estimated marginal effect of the minimum wage on the standard deviation of log earnings (``St.d.'' on the x-axis) and on wages between the 10th and the 90th percentiles of the wage distribution (``10'' to ``90'' on the x-axis) relative to some base wage $p$ are shown. Panel \subref{fig: Lee data model vs data alternative specs A} uses the 50th percentile as the base wage (i.e., $p=50$), while panel \subref{fig: Lee data model vs data alternative specs B} uses the 90th percentile as the base wage (i.e., $p=90$). The four error bars and four shaded areas represent 99 percent confidence intervals based on regular (i.e., not clustered) standard errors. %
      \emph{\scriptsize{}Source: } RAIS, 1996--2018, and model.}
    \end{spacing}
  \end{minipage}
  %
\end{figure}



\clearpage
\subsection{AKM decomposition\label{app_subsec:akm_decomposition}}
Table \ref{table: AKM two periods} summarizes the impact of the minimum wage on earnings inequality as viewed through the lens of the AKM decomposition. The increase in the minimum wage accounts for about a third of the fall in the overall variance of earnings over this period in Brazil. It accounts for roughly half of the compression in the variance in AKM person fixed effects and 10 percent of the fall in the variance of the AKM firm effects. We caution, however, that \textit{all} of the fall in inequality in the model is due to changes in firms' wage and vacancy policies.

\begin{table}[!htb]
  %
  \centering
  \caption{Impact of minimum wage on AKM decomposition, model versus data\label{table: AKM two periods}}
  %
  \resizebox{\textwidth}{!}{%
  \begin{tabular}{l cc c cc c ccc}
  \hline \hline \addlinespace[1ex]
  & \multicolumn{2}{c}{1994--1998} && \multicolumn{2}{c}{2014--2018} && \multicolumn{3}{c}{Change} \\ \cline{2-3} \cline{5-6} \cline{8-10}
   & \phantom{\textbf{Due to M}} & \phantom{\textbf{Due to M}} && \phantom{\textbf{Due to M}} & \phantom{\textbf{Due to M}} && \phantom{\textbf{Due to M}} & \phantom{\textbf{Due to M}} & \phantom{\textbf{Due to}} \\ \addlinespace[-1ex]
  & Data & Model && Data & Model && Data & Model & \textbf{Due to MW} \\ \hline\addlinespace[1.5ex]
  Variance of log wages & 0.704 & 0.600 && 0.436 & 0.478 && -0.268 & -0.121 & \textbf{45.3\%} \\
  Variance of AKM person FEs & 0.333 & 0.259 && 0.262 & 0.208 && -0.070 & -0.051 & \textbf{72.7\%} \\
  Variance of AKM firm FEs & 0.217 & 0.195 && 0.085 & 0.176 && -0.131 & -0.018 & \textbf{14.1\%} \\
  Variance of AKM residual & 0.032 & 0.035 && 0.016 & 0.036 && -0.016 & 0.001 & \textbf{-8.1\%} \\
  2$\times$covariance AKM person-firm FEs & 0.123 & 0.111 && 0.072 & 0.058 && -0.051 & -0.053 & \textbf{104.3\%} \\
  \addlinespace[.5ex] \hline
  \end{tabular}
}

  %
  \posttabvspace
  %
  \begin{minipage}[t]{1\columnwidth}%
    \begin{spacing}{0.75}
      \emph{\scriptsize{}Notes: }{\scriptsize{}Men aged 18--54. Estimated impact of a 57.7 log point increase in the minimum wage in the model as well as the raw data. Model and data sample selection and variable construction is identical. %
      \emph{\scriptsize{}Source: } Model and RAIS 1994--2018.}
    \end{spacing}
  \end{minipage}
  %
\end{table}



\clearpage
\subsection{Young workers only\label{app_subsec:young_only}}

This section reestimates the model for young workers aged 18--36. In doing so, we are able to compare the predicted effects of the minimum wage for this population subgroups and compare the predicted effects to those of the full population of workers aged 18--54. To this end, we reproduce all target moments for this subpopulation and recompute the set of model parameters to match these moments. With the recalibrated model in hand, we resimulate the effects of the same minimum wage increase as we considered for the full population.

In the end, we reach broadly similar conclusions for the set of young workers, shown in Table \ref{tab_young}, as for the full population, shown in Table \ref{table: impact of minimum wage} of the main text. While the differences between the two subpopulations are not too striking, the estimated effects are, if anything, slightly more pronounced in the population of young workers. The finding that the results are so similar is in turn driven by the fact that the targeted moments do not change that much between the young subpopulation and the overall population, and hence the estimated parameter values do not change that much.

\begin{table}[!htb]
  \begin{small}
    \centering
    \caption{Total impact of the minimum wage on wage inequality for the subpopulation of young workers aged 18--36, model versus data\label{tab_young}}
    %
    \begin{tabular}{l cc c cc c ccc}
  \hline \hline
  & \multicolumn{2}{c}{1994--1998} && \multicolumn{2}{c}{2010--2014} && \multicolumn{3}{c}{Change} \\ \cline{2-3} \cline{5-6} \cline{8-10}
  & Data & Model && Data & Model && Data & Model & \textbf{Due to MW} \\ \hline
  Variance & 0.578 & 0.439 && 0.338 & 0.312 && -0.240 & -0.127 & \textbf{52.9\%} \\
  P5-50 & -1.007 & -1.020 && -0.538 & -0.678 && 0.469 & 0.342 & \textbf{72.9\%} \\
  P10-50 & -0.825 & -0.819 && -0.471 & -0.598 && 0.354 & 0.220 & \textbf{62.2\%} \\
  P25-50 & -0.449 & -0.456 && -0.275 & -0.350 && 0.174 & 0.107 & \textbf{61.2\%} \\
  P75-50 & 0.538 & 0.508 && 0.390 & 0.445 && -0.147 & -0.063 & \textbf{43.0\%} \\
  P90-50 & 1.131 & 0.990 && 0.897 & 0.893 && -0.234 & -0.097 & \textbf{41.6\%} \\
  P95-50 & 1.525 & 1.253 && 1.270 & 1.145 && -0.255 & -0.108 & \textbf{42.3\%} \\  \hline
\end{tabular}

    %
    \begin{minipage}[t]{1\columnwidth}%
      \begin{spacing}{0.75}
        {\scriptsize \textit{Notes:} Table shows estimated impact of a 57.7 log point increase in the minimum wage in the model as well as the raw data for the sample of young workers (aged 18--36). Percentile ratios of log wages, constructed as the sum of wages from a given employer over the five year sample period divided by the sum of months worked for that employer over each five year period. Model and data sample selection and variable construction is identical. See text for detail. %
        \textit{Source:} Model and RAIS.}
      \end{spacing}
    \end{minipage}
  %
\end{small}
\end{table}




\clearpage
\subsection{Empirical support for the impact of the minimum wage on sorting\label{app_subsec:impact_sorting}}

Figure \ref{figure: changes in sorting} illustrates this change in sorting in the model and provides empirical support for it. Panel \subref{figure: changes in sorting a} plots average firm productivity by worker ability, highlighting that average productivity rises, particularly among the lowest-ability workers. The reason is that matches between low-ability workers and low-productivity firms become unviable when the minimum wage is raised. Panel \subref{figure: changes in sorting b} of the figure provides reduced-form support consistent with this prediction. Specifically, it plots average AKM firm fixed effect by decile of AKM worker fixed effects in the model and data. For completeness, we replicate the level in the initial period from Figure \ref{figure: model validation}. As the minimum wage is increased, the average AKM firm fixed effect rises disproportionately among the lowest AKM worker fixed effects workers.

\begin{figure}[!htb]
  %
  \centering
  \caption{Reallocation of lower-ability workers toward higher-productivity firms, model versus data \label{figure: changes in sorting}}
  %
  \prefigvspace
  %
  \subfloat[Average productivity by worker ability \label{figure: changes in sorting a}]{\includegraphics[trim={.1in .12in 0in .3in},clip,width=.40\columnwidth]{_figures/figE2A.png}}% _figures/Change_productivity_model.png
  \subfloat[Average AKM firm FE by person FEs \label{figure: changes in sorting b}]{\includegraphics[trim={.1in .12in 0in .3in},clip,width=.40\columnwidth]{_figures/figE2B.png}}% _figures/ModelDataChange_fe_mean.png
  %
  \\
  %
  \postfigvspace
  %
  \begin{minipage}[t]{1\columnwidth}%
    \begin{spacing}{0.75}
      \emph{\scriptsize{}Notes: }{\scriptsize{}Men aged 18--54. Estimated impact of a 57.7 log point increase in the productivity adjusted real minimum wage. Panel A shows average log firm productivity by worker ability, $\int_{\underline{z}}^{\overline{z}} \log z dG(z|a,s)$, in market for workers with $s(a)>0$ (the vast majority of workers). Panel B shows average AKM firm fixed effect by decile of AKM worker fixed effects. AKM regression is estimated on model-simulated monthly data aggregated to the annual level following an identical sample selection and variable construction methodology as in the data. %
      \emph{\scriptsize{}Source: } Model and RAIS 1994--2018.}
    \end{spacing}
  \end{minipage}
  %
\end{figure}




\clearpage
\subsection{Further results on heterogeneity in effects on disemployment and firm size\label{app_subsec:employment_effects_het_and_fsize}}

Figure \ref{figure: impact of minimum wage on aggregates} sheds light on the heterogeneous effects of the minimum wage on employment and firm sizes. Panel \subref{figure: impact of minimum wage on aggregates a} shows that, among the lowest-ability workers, employment falls by over 13 percent, while employment is essentially unaffected among workers above the bottom third of the ability distribution. Panel \subref{figure: impact of minimum wage on aggregates b} focuses again on a group of workers most affected by the minimum wage---specifically, the first percentile of worker ability. Firms near the bottom of the firm productivity distribution shrink by almost 30 percent, while firms in the top five percent of the productivity distribution in fact expand in response to an increase in the minimum wage, for reasons that we analyze further below.


\begin{figure}[!htb]
  %
  \centering
  \caption{Impact of minimum wage on aggregate outcomes, model\label{figure: impact of minimum wage on aggregates}}
  %
  \prefigvspace
  %
  \subfloat[Change in employment by worker ability\label{figure: impact of minimum wage on aggregates a}]{\includegraphics[trim={.1in .12in 0in .3in},clip,width=.40\linewidth]{_figures/figE3A.png}}% _figures/MWandEmployment.png
  \subfloat[Change in size by firm productivity\label{figure: impact of minimum wage on aggregates b}]{\includegraphics[trim={.1in .12in 0in .3in},clip,width=.40\linewidth]{_figures/figE3B.png}}% _figures/MWandSize.png
  %
  \\
  %
  \postfigvspace
  %
  \begin{minipage}[t]{1\columnwidth}%
    \begin{spacing}{0.75}
      {\scriptsize \textit{Notes:} Impact of a 57.7 log point increase in the minimum wage in the estimated model. Panel \subref{figure: impact of minimum wage on aggregates a} shows the log change in employment rate by worker ability in market with positive search efficiency, $s(a)>0$ (the vast majority of workers). Panel \subref{figure: impact of minimum wage on aggregates b} shows the log change in employment by firms ranked by employment-unweighted productivity in the first percentile of the worker ability distribution among workers with $s(a)>0$. %
      \textit{Source:} Model.}
    \end{spacing}
  \end{minipage}
  %
\end{figure}




\clearpage
\subsection{Robustness of the small employment response\label{app_subsec:emp_effects_dependence_on_parameters}}

In light of our finding of modest aggregate disemployment effects of the minimum wage, we now investigate the robustness of our conclusions. In order for us to find a larger aggregate employment response, one or more of the channels in our decomposition based on equation \eqref{eq: employment decomposition} would need to be larger in magnitude. Their magnitude in turn relates to a set of estimated parameters and equilibrium objects, including the job finding rate, $p(a,s)$, the separation rate $\delta(a,s)$, the on-the-job search efficiency, $s(a)$, the elasticity of the vacancy cost function, $\eta$, and the elasticity of the matching function, $\alpha$. We start by discussing the parameters that most clearly stand out based on \eqref{eq: employment decomposition}, and subsequently present the sensitivity analysis with respect to the other model parameters.

Compared to common values for the U.S., we estimate a relatively low job finding rate, $p(a,s)$, and a comparable separation rate $\delta(a,s)$ for Brazil. All else equal, this would lead to a smaller job finding channel in the U.S. than what we find in Brazil. Compared to the U.S., we estimate a relatively high on-the-job search efficiency, $s(a)$. To see that this high value has limited effects on our results, consider the opposite extreme with $s(a) \approx 0$. Then the congestion channel approximately equals $\alpha/(1 - 0.4\alpha) = 0.6$. Hence, even in this extreme case, the congestion channel is a significant moderation force.

The sensitivity of the aggregate employment response with respect to the elasticity of the vacancy cost function, $\eta$, and the elasticity of the matching function, $\alpha$, are shown in Figure \ref{figure: robustness employment}.%
%
Panel \subref{figure: robustness employment a} shows that, perhaps surprisingly, a higher value of $\eta$ actually amplifies the disemployment effects of the minimum wage. This is due to the equilibrium reallocation effects. Under a higher value of $\eta$, the minimum wage leads to smaller employment cuts among unproductive firms but also reduces the scaling up of more productive firms. Quantitatively, the latter force outweighs the former.

Panel \subref{figure: robustness employment b} shows that higher values of $\lambda$ are associated with greater disemloyment effects. But our preset value of $\alpha = 0.50$ is relatively high compared to other values in the literature. For example, \citetappendix{MeghirNarita2015_appendix} estimate $\alpha = 0.34$ using the Brazil's PME data, \citetappendix{shimer2005_appendix} estimates $\alpha=0.28$ for the U.S., and \citetappendix{mortensennagypal2007_appendix} argue that a reasonable value is $\alpha=0.40$. Therefore, our results likely represent an upper bound on the aggregate disemployment effects of the minimum wage.


\begin{figure}[!htb]
  %
  \centering
  \caption{Sensitivity of the employment effect of the minimum wage to parameter values\label{figure: robustness employment}}
  %
  \prefigvspace
  %
  \subfloat[Curvature of vacancy cost, $\eta$\label{figure: robustness employment a}]{\includegraphics[trim={0.0in .12in 0in .3in},clip,width=.45\linewidth]{_figures/figE4A.png}}% _figures/Robustness_employment_eta.png
  \subfloat[Elasticity of matches w.r.t. vacancies, $\alpha$\label{figure: robustness employment b}]{\includegraphics[trim={0.0in .12in 0in .3in},clip,width=.45\linewidth]{_figures/figE4B.png}}% _figures/Robustness_employment_alpha.png
  %
  \\
  %
  \postfigvspace
  %
  \begin{minipage}[t]{1\columnwidth}%
    \begin{spacing}{0.75}
      \emph{\scriptsize{}Notes: }{\scriptsize{} Estimated impact of a 57.7 log point increase in the minimum wage across different parameter values, varying one parameter at a time and holding fixed all other parameters at their estimated values.
      \emph{\scriptsize{}Source: } Model.}
    \end{spacing}
  \end{minipage}
  %
\end{figure}



Figures \ref{figure: robustness_appendix3}--\ref{figure: robustness_appendix5} conduct the same exercise as above across the remaining 11 internally estimated structural parameters of the model, as well as for the calibrated job finding rate $\lambda$ and separation rate of minimum wage workers, $\delta_{MW}$. As for the impact of the minimum wage on inequality, the key parameters determining the employment effect of the minimum wage are the mean of worker ability ($\mu$), the tail index of the firm productivity distribution ($\zeta$), the slope of the reservation wage ($r_1$), and to a lesser extent the job finding rate ($\lambda$). The larger is $\mu$, the less binding is the minimum wage initially and the smaller is the effect of an increase in the minimum wage on employment (as well as inequality---recall Figure \ref{figure: robustness}). A larger $\zeta$ (i.e. a thinner tail of the firm productivity distribution) raises the disemployment effect of the minimum wage, as there is a larger number of low productive firms that are heavily exposed to the minimum wage and fewer high productive firms to pick up the employment slack. The faster the reservation wage rises in ability---the larger is $r_1$---the less the minimum wage binds and hence the smaller is the disemployment effect of a rise in the minimum wage. A lower $\lambda$ is associated with a smaller disemployment effect of the minimum wage. The remaining parameters have at most a modest effect on the impact of a rise in the minimum wage on employment.

\begin{figure}[!htb]
  %
  \centering
  \caption{Change in the employment rate across model parameters\label{figure: robustness_appendix3}}
  %
  \prefigvspace
  %
  \hspace*{\fill}%
  \csubfloat[$\mu$\label{figure: robustness_appendix3 A}]{%
   \includegraphics[trim={.0in .12in 0in .3in},clip,width=.40\columnwidth]{_figures/figE5A.png}% _figures/Robustness_employment_mu.png
  %
  }\centerhfill[\qquad\qquad\qquad\qquad\qquad]
  \csubfloat[$\sigma$\label{figure: robustness_appendix3 B}]{%
   \includegraphics[trim={.0in .12in 0in .3in},clip,width=.40\columnwidth]{_figures/figE5B.png}% _figures/Robustness_employment_sigma.png
  %
  }\hspace*{\fill}

  \hspace*{\fill}%
  \csubfloat[$\zeta$\label{figure: robustness_appendix3 C}]{%
   \includegraphics[trim={0in .12in 0in .3in},clip,width=.40\columnwidth]{_figures/figE5C.png}% _figures/Robustness_employment_zeta.png
  %
  }\centerhfill[\qquad\qquad\qquad\qquad\qquad]
  \csubfloat[$\varepsilon$\label{figure: robustness_appendix3 D}]{%
   \includegraphics[trim={0in .12in 0in .3in},clip,width=.40\columnwidth]{_figures/figE5D.png}% _figures/Robustness_employment_error.png
  %
  }\hspace*{\fill}

  \hspace*{\fill}%
  \csubfloat[$\delta_0$\label{figure: robustness_appendix3 E}]{%
   \includegraphics[trim={0in .12in 0in .3in},clip,width=.40\columnwidth]{_figures/figE5E.png}% _figures/Robustness_employment_delta0.png
  %
  }\centerhfill[\qquad\qquad\qquad\qquad\qquad]
  \csubfloat[$\delta_1$\label{figure: robustness_appendix3 F}]{%
   \includegraphics[trim={0in .12in 0in .3in},clip,width=.40\columnwidth]{_figures/figE5F.png}% _figures/Robustness_employment_delta1.png
  %
  }\hspace*{\fill}
  %
  \\
  %
  \postfigvspace
  %
  \begin{minipage}[t]{1\columnwidth}%
    \begin{spacing}{0.75}
      \emph{\scriptsize{}Notes: }{\scriptsize{}Estimated impact of a 57.7 log point increase in the minimum wage across different parameter values, varying one parameter at a time and holding fixed all other parameters at their estimated values.
      \emph{\scriptsize{}Source: } Model.}
    \end{spacing}
  \end{minipage}
  %
\end{figure}

\begin{figure}[!htb]
  %
  \centering
  \caption{Change in the employment rate across model parameters, continued\label{figure: robustness_appendix4}}
  %
  \prefigvspace
  %
  \hspace*{\fill}%
  \csubfloat[$\phi_0$\label{figure: robustness_appendix4 A}]{%
   \includegraphics[trim={0in .12in 0in .3in},clip,width=.40\columnwidth]{_figures/figE6A.png}% _figures/Robustness_employment_phi0.png
  %
  }\centerhfill[\qquad\qquad\qquad\qquad\qquad]
  \csubfloat[$\phi_1$\label{figure: robustness_appendix4 B}]{%
   \includegraphics[trim={0in .12in 0in .3in},clip,width=.40\columnwidth]{_figures/figE6B.png}% _figures/Robustness_employment_phi1.png
  %
  }\hspace*{\fill}
  %
  \\
  %
  \hspace*{\fill}%
  \csubfloat[$r_0$\label{figure: robustness_appendix4 C}]{%
   \includegraphics[trim={0in .12in 0in .3in},clip,width=.40\columnwidth]{_figures/figE6C.png}% _figures/Robustness_employment_r0.png
  %
  }\centerhfill[\qquad\qquad\qquad\qquad\qquad]
  \csubfloat[$r_1$\label{figure: robustness_appendix4 D}]{%
   \includegraphics[trim={0in .12in 0in .3in},clip,width=.40\columnwidth]{_figures/figE6D.png}% _figures/Robustness_employment_r1.png
  %
  }\hspace*{\fill}

  \hspace*{\fill}%
  \csubfloat[$\pi$\label{figure: robustness_appendix4 E}]{%
   \includegraphics[trim={0in .12in 0in .3in},clip,width=.40\columnwidth]{_figures/figE6E.png}% _figures/Robustness_employment_pi.png
    }\centerhfill[\qquad\qquad\qquad\qquad\qquad]
  \csubfloat[$\lambda$\label{figure: robustness_appendix4 F}]{%
   \includegraphics[trim={0in .12in 0in .3in},clip,width=.40\columnwidth]{_figures/figE6F.png}% _figures/Robustness_employment_lambda.png
  %
  }\hspace*{\fill}
  %
  \\
  %
  \postfigvspace
  %
  \begin{minipage}[t]{1\columnwidth}%
    \begin{spacing}{0.75}
      \emph{\scriptsize{}Notes: }{\scriptsize{}Estimated impact of a 57.7 log point increase in the minimum wage across different parameter values, varying one parameter at a time and holding fixed all other parameters at their estimated values.
      \emph{\scriptsize{}Source: } Model.}
    \end{spacing}
  \end{minipage}
  %
\end{figure}

\begin{figure}[!htb]
  %
  \centering
  \caption{Change in the employment rate across model parameters, continued\label{figure: robustness_appendix5}}
  %
  \prefigvspace
  %
  \hspace*{\fill}%
  \csubfloat[$\delta_{MW}$\label{figure: robustness_appendix5 A}]{%
   \includegraphics[trim={0in .12in 0in .3in},clip,width=.40\columnwidth]{_figures/figE7.png}% _figures/Robustness_employment_deltaMW.png
  %
  }\hspace*{\fill}
  %
  \\
  %
  \postfigvspace
  %
  \begin{minipage}[t]{1\columnwidth}%
    \begin{spacing}{0.75}
      \emph{\scriptsize{}Notes: }{\scriptsize{}Estimated impact of a 57.7 log point increase in the minimum wage across different parameter values, varying one parameter at a time and holding fixed all other parameters at their estimated values.
      \emph{\scriptsize{}Source: } Model.}
    \end{spacing}
  \end{minipage}
  %
\end{figure}




\clearpage
\subsection{Dependence of estimated inequality effect on model parameters\label{app_subsec:effects_dependence_on_parameters}}

Figures \ref{figure: robustness_appendix}--\ref{figure: robustness_appendix2} conduct the same exercise as in Section \ref{subsec:comparative_statics} across the remaining nine internally estimated structural parameters of the model, the calibrated separation rate of minimum wage workers ($\delta_{MW}$) and the preset elasticity of the matching function ($\alpha$). Most of these parameters have at most a modest effect on the impact of a rise in the minimum wage on inequality. The main exception is the slope of the reservation wage, $r_1$. Intuitively, a higher reservation wage leaves less scope for the minimum wage to impact markets, as the reservation increasingly becomes the binding constraints across worker ability markets. Recall from Appendix \ref{app_subsec:identification} that the parameter the model primarily struggles to inform well based on the available data is $r_0$. This parameter, however, is not critical in terms of driving the estimated impact of the minimum wage on inequality. Finally, $\delta_{MW}$ and $\alpha$ have no meaningful effect on the impact of the minimum wage on inequality.

\begin{figure}[!htb]
  %
  \centering
  \caption{Change in the variance of log wages across model parameters\label{figure: robustness_appendix}}
  %
  \prefigvspace
  %
  \hspace*{\fill}%
  \csubfloat[$\sigma$\label{figure: robustness_appendix A}]{%
   \includegraphics[trim={.0in .12in 0in .3in},clip,width=.40\columnwidth]{_figures/figE8A.png}% _figures/Robustness_wage_sigma.png
  %
  }\centerhfill[\qquad\qquad\qquad\qquad\qquad]
  \csubfloat[$\eta$\label{figure: robustness_appendix B}]{%
   \includegraphics[trim={.0in .12in 0in .3in},clip,width=.40\columnwidth]{_figures/figE8B.png}% _figures/Robustness_wage_eta.png
  %
  }\hspace*{\fill}

  \hspace*{\fill}%
  \csubfloat[$\varepsilon$\label{figure: robustness_appendix C}]{%
   \includegraphics[trim={0in .12in 0in .3in},clip,width=.40\columnwidth]{_figures/figE8C.png}% _figures/Robustness_wage_error.png
  %
  }\centerhfill[\qquad\qquad\qquad\qquad\qquad]
  \csubfloat[$\delta_1$\label{figure: robustness_appendix D}]{%
   \includegraphics[trim={0in .12in 0in .3in},clip,width=.40\columnwidth]{_figures/figE8D.png}% _figures/Robustness_wage_delta1.png
  %
  }\hspace*{\fill}

  \hspace*{\fill}%
  \csubfloat[$\phi_0$\label{figure: robustness_appendix E}]{%
   \includegraphics[trim={0in .12in 0in .3in},clip,width=.40\columnwidth]{_figures/figE8E.png}% _figures/Robustness_wage_phi0.png
  %
  }\centerhfill[\qquad\qquad\qquad\qquad\qquad]
  \csubfloat[$\phi_1$\label{figure: robustness_appendix F}]{%
   \includegraphics[trim={0in .12in 0in .3in},clip,width=.40\columnwidth]{_figures/figE8F.png}% _figures/Robustness_wage_phi1.png
  %
  }\hspace*{\fill}
  %
  \\
  %
  \postfigvspace
  %
  \begin{minipage}[t]{1\columnwidth}%
    \begin{spacing}{0.75}
      \emph{\scriptsize{}Notes: }{\scriptsize{}Estimated impact of a 57.7 log point increase in the minimum wage across different parameter values, varying one parameter at a time and holding fixed all other parameters at their estimated values.
      \emph{\scriptsize{}Source: } Model.}
    \end{spacing}
  \end{minipage}
  %
\end{figure}


\begin{figure}[!htb]
  %
  \centering
  \caption{Change in the variance of log wages across model parameters, continued\label{figure: robustness_appendix2}}
  %
  \prefigvspace
  %
  \hspace*{\fill}%
  \csubfloat[$r_0$\label{figure: robustness_appendix2 A}]{%
   \includegraphics[trim={0in .12in 0in .3in},clip,width=.40\columnwidth]{_figures/figE9A.png}% _figures/Robustness_wage_r0.png
  %
  }\centerhfill[\qquad\qquad\qquad\qquad\qquad]
  \csubfloat[$r_1$\label{figure: robustness_appendix2 B}]{%
   \includegraphics[trim={0in .12in 0in .3in},clip,width=.40\columnwidth]{_figures/figE9B.png}% _figures/Robustness_wage_r1.png
  %
  }\hspace*{\fill}
  %
  \\
  %
  \hspace*{\fill}%
  \csubfloat[$\pi$\label{figure: robustness_appendix2 C}]{%
   \includegraphics[trim={0in .12in 0in .3in},clip,width=.40\columnwidth]{_figures/figE9C.png}% _figures/Robustness_wage_pi.png
  %
  }\centerhfill[\qquad\qquad\qquad\qquad\qquad]
  \csubfloat[$\delta_{MW}$\label{figure: robustness_appendix2 D}]{%
   \includegraphics[trim={0in .12in 0in .3in},clip,width=.40\columnwidth]{_figures/figE9D.png}% _figures/Robustness_wage_deltaMW.png
  %
  }\hspace*{\fill}

  \hspace*{\fill}%
  \csubfloat[$\alpha$\label{figure: robustness_appendix2 E}]{%
   \includegraphics[trim={0in .12in 0in .3in},clip,width=.40\columnwidth]{_figures/figE9E.png}% _figures/Robustness_wage_alpha.png
  }\hspace*{\fill}
  %
  \\
  %
  \postfigvspace
  %
  \begin{minipage}[t]{1\columnwidth}%
    \begin{spacing}{0.75}
      \emph{\scriptsize{}Notes: }{\scriptsize{}Estimated impact of a 57.7 log point increase in the minimum wage across different parameter values, varying one parameter at a time and holding fixed all other parameters at their estimated values.
      \emph{\scriptsize{}Source: } Model.}
    \end{spacing}
  \end{minipage}
  %
\end{figure}
