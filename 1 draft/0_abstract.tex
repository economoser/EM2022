% !TEX root = EIMW2022.tex

\begin{abstract}
  We show that a 128 percent real increase in the minimum wage accounts for a large decline in earnings inequality in Brazil between 1996 and 2018. To this end, we combine administrative and survey data with an equilibrium model of the Brazilian labor market. Our results imply that the minimum wage has far-reaching spillover effects on wages higher up in the distribution, accounting for 45 percent of a large fall in earnings inequality over this period. At the same time, the effects of the minimum wage on employment and output are muted by reallocation of workers toward more productive firms.
\end{abstract}
